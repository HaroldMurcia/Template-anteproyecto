\documentclass[../Main.tex]{subfiles}
\begin{document}
Formular un solo objetivo general, coherente con el problema a investigar. La formulación de
objetivos claros y viables constituye una base importante para juzgar el resto de la propuesta y,
además, facilita la estructuración de la metodología. Los objetivos deben corresponderse con el
problema o situación a tratar. Los objetivos, tanto el general como los específicos, deben empezar
con un verbo en infinitivo que denote una acción concreta, medible y verificable. Deben evitarse
verbos abstractos como aprender, conocer, ver, idear, pensar, etc.




\end{document}
